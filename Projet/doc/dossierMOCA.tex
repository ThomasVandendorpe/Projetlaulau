\documentclass{report}
\usepackage[english]{babel}
\usepackage[utf8]{inputenc}
\usepackage[T1]{fontenc}
\usepackage{listings}
\usepackage{titlesec}
 
\titleformat{\chapter}[display]
  {\normalfont\bfseries}{}{0pt}{\Huge}

\title{Méthodes et Outils pour la Conception Avancée}
\author{Castel Antonin, Reboul Paul, Vandendorpe Thomas}
\begin{document}
\maketitle{}
\tableofcontents

\chapter{Introduction}

Ce dossier rassemble les travaux effectués lors du cours de ``Méthodes et Outils pour la Conception Avancée''. L'objectif ici, est de montrer les bonnes pratiques à avoir lors de la conception d'un programme. Nous illustrerons ces bonnes pratiques à partir de l'exemple d'un programme de puissance 4 (en langage C), initialement codé de façon peu rigoureuse, que nous tenterons d'améliorer tout au long de ce dossier. Par améliorer, on entend  la maintenabilité, la réutilisabilité et la documentation du code, la qualité et la couverture des tests, la détection de défauts et d'erreurs, l'analyse et l'amélioration des performances, ainsi que l'analyse de la vulnérabilité à certaines attaques. Nous expliquerons les différentes méthodes et outils permettants ces améliorations...

\chapter{Modularité, maintenabilité, réutilisabilité}
\section{Titre}
...
\section{Titre}
...
\end{document}
